
%%% Titulní strana práce

\pagestyle{empty}
\begin{center}

\large

Univerzita Karlova v Praze

\medskip

Matematicko-fyzikální fakulta

\vfill

{\bf\Large BAKALÁŘSKÁ PRÁCE}

\vfill

\centerline{\mbox{\includegraphics[width=60mm]{img/logo.eps}}}

\vfill
\vspace{5mm}

{\LARGE David Marek}

\vspace{15mm}

% Název práce přesně podle zadání
{\LARGE\bfseries Řízení robota e-puck v Pythonu}

\vfill

% Název katedry nebo ústavu, kde byla práce oficiálně zadána
% (dle Organizační struktury MFF UK)
Kabinet software a výuky informatiky

\vfill

\begin{tabular}{rl}

Vedoucí bakalářské práce: & RNDr. František Mráz, CSc.\\
\noalign{\vspace{2mm}}
Studijní program: & Informatika \\
\noalign{\vspace{2mm}}
Studijní obor: & Obecná Informatika \\
\end{tabular}

\vfill

% Zde doplňte rok
Praha 2011

\end{center}

\newpage

%%% Následuje vevázaný list -- kopie podepsaného "Zadání bakalářské práce".
%%% Toto zadání NENÍ součástí elektronické verze práce, nescanovat.

%%% Na tomto místě mohou být napsána případná poděkování (vedoucímu práce,
%%% konzultantovi, tomu, kdo zapůjčil software, literaturu apod.)

\openright

\noindent
\begin{otherlanguage}{czech}
Rád bych poděkoval vedoucímu bakalářské práce RNDr. Františku Mrázovi za
odborné vedení, cenné rady a zapůjčení e-puck robota.
\end{otherlanguage}

\newpage

%%% Strana s čestným prohlášením k bakalářské práci

\vglue 0pt plus 1fill

\noindent
Prohlašuji, že jsem tuto bakalářskou práci vypracoval samostatně a výhradně s
použitím citovaných pramenů, literatury a dalších odborných zdrojů.

\medskip\noindent
Beru na vědomí, že se na moji práci vztahují práva a povinnosti vyplývající ze
zákona č. 121/2000 Sb., autorského zákona v platném znění, zejména skutečnost,
že Univerzita Karlova v Praze má právo na uzavření licenční smlouvy o~užití
této práce jako školního díla podle \S 60 odst. 1 autorského zákona.

\vspace{10mm}

\hbox{\hbox to 0.5\hsize{%
V ........ dne ............
\hss}\hbox to 0.5\hsize{%
podpis
\hss}}

\vspace{20mm}
\newpage

%%% Povinná informační strana bakalářské práce

\vbox to 0.5\vsize{
\setlength\parindent{0mm}
\setlength\parskip{5mm}

\begin{otherlanguage}{czech}

Název práce:
Řízení robota e-puck v Pythonu\\
% přesně dle zadání
Autor:
David Marek\\
Katedra:  % Případně Ústav:
Kabinet software a výuky informatiky\\
% dle Organizační struktury MFF UK
Vedoucí bakalářské práce:
RNDr. František Mráz, CSc., Kabinet software a výuky informatiky
% dle Organizační struktury MFF UK, případně plný název pracoviště mimo MFF UK

Abstrakt:
V předložené práci studujeme ovládání miniaturního e-puck robota. Pro
programování robota jsme zvolili programovací jazyk Python, díky možnostem
rychlého vývoje a interaktivního ovládání. Zaměřujeme se na dálkové ovládání
robota z uživatelského počítače pomocí Bluetooth a na mechanismy, které zajistí
robustní komunikaci. Uživatel knihovny dostává na výběr, zda-li si zvolí
synchronní anebo asynchronní komunikaci. Při vývoji synchronní komunikace jsme
dávali důraz na interaktivitu a transparentnost. Při asynchronní komunikaci
klademe důraz na spolehlivost. Dále jsme vytvořili několik ukázkových programů,
které ilustrují použití knihovny.
% abstrakt v rozsahu 80-200 slov; nejedná se však o opis zadání bakalářské
% práce

Klíčová slova:
řízení robota, Python, e-puck
% 3 až 5 klíčových slov

\end{otherlanguage}

\vss}\nobreak\vbox to 0.49\vsize{
\setlength\parindent{0mm}
\setlength\parskip{5mm}

Title:
Controlling e-puck robot in Python \\
% přesný překlad názvu práce v angličtině
Author:
David Marek \\
Department:
Department of Software and Computer Science Education \\
% dle Organizační struktury MFF UK v angličtině
Supervisor:
RNDr. František Mráz, CSc., Department of Software and Computer Science Education

% dle Organizační struktury MFF UK, případně plný název pracoviště
% mimo MFF UK v angličtině
Abstract:
In the present work we study controlling a miniature e-puck robot. We have
chosen to use the Python programming language because it is very easy to use
for rapid development and interactive control. We focus on how to control the
robot remotely from PC via Bluetooth. An important aspect of controlling the
robot remotely is the mechanism enabling robust communication. The user can
choose between synchronous and asynchronous communication. The former is
primarily used for interactive controlling because of its transparency. The
latter is a right choice for reliable communication. The work also contains a
few example programs to present how to use the library.
% abstrakt v rozsahu 80-200 slov v angličtině; nejedná se však o překlad
% zadání bakalářské práce

Keywords:
robot programming, robot control, Python, e-puck
% 3 až 5 klíčových slov v angličtině

\vss}

\newpage
