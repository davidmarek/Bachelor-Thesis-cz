\chapter{Implementace}
\label{Implementace}

    Hlavní součástí této práce je samozřejmě samotná implementace knihovny.
    Důležitou změnou je úprava firmware robota. V sekci \ref{btcom} je
    představen firmware BTcomDM, který je používán pro ovládání robota. Jsou
    zde také uvedeny změny, které bylo nutné provést pro zabezpečení spolehlivé
    asynchronní komunikace.

    Pro rychlé testování příkazů a interaktivní ovládání robota je určena
    synchronní komunikace. V kapitole \ref{btcom:sync} je popsáno, jak
    synchronní komunikace vypadá a~k~čemu všemu může sloužit.

    Implementace asynchronní komunikace je představena v sekci
    \ref{async-impl}. Jedná se o představení mechanismů, díky kterým je možné
    provádět komunikaci asynchronně. K asynchronní komunikaci také patří
    zpracování odpovědi a znovu-odeslání ztracených příkazů.

    Nakonec je představena pro uživatele ta nejzajímavější část a to rozhraní
    pro komunikaci s robotem (sekce \ref{controller}). Jedná se o most mezi
    uživatelem a třídami, které se starají o komunikaci. Dochází zde k převodu
    příkazů do srozumitelné formy pro robota a naopak také k převádění odpovědí
    od robota do formátu čitelného pro uživatele.

    \section{BTcomDM firmware}
    \label{btcom}

    Robot e-puck v sobě obsahuje programovatelný mikroprocesor. Většinou se
    vývoj programu pro robota omezuje na psaní programu právě pro tento
    mikroprocesor. Takový program je pak ovšem, co se výpočetního výkonu týče,
    velmi limitován. My se ovšem od tohoto způsobu vývoje odkloníme a budeme
    používat dálkové ovládání. Programátor píše program, který běží na PC, a
    pouze jednotlivé příkazy jsou posílány do robota ke zpracování.

    Je tedy potřeba, aby v robotovi byl software, který dokáže přijímat
    příkazy, zpracovávat je a odesílat odpovědi. K tomuto účelu byl upraven
    tvůrci robota poskytovaný firmware BTcom. Jedná se o jednoduchý program,
    který společně se standardní knihovnou dodávanou k e-puck robotovi
    zpřístupňuje všechny senzory a akční členy robota pomocí jednoduchého
    protokolu přes Bluetooth rozhraní. Upravená verze BTcomDM není zpětně
    kompatibilní s původním BTcom a tedy programy napsané pro původní verzi
    firmware nemohou využívat výhod upravené verze.

    \subsection{Příkazy}
    \label{btcom:prikazy}

    Příkazy je možné posílat v textovém nebo binárním režimu. Textový režim
    slouží k posílání jednoduchých dat jako je ovládání led diod nebo nastavení
    rychlosti motorů. Binární režim slouží k získávání větších dat, jako jsou
    například snímky z kamery.

    Příklad jak může vypadat příkaz poslaný robotovi: {\tt
    Da,100,100\textbackslash r}. Zde se jedná o textový příkaz, první znak
    určuje typ příkazu, písmeno \uv{D} je příkaz pro změnu rychlosti kol. Druhé
    písmeno je tzv. timestamp, jedná se právě o kontrolní znak, který určuje
    pořadí příkazů a odlišuje příkazy stejného typu. Dále následují dvě čísla,
    první určuje rychlost levého kola, druhé rychlost pravého kola. Nakonec je
    znak konce řádku, který určuje konec příkazu.

    Binární příkazy se také rozlišují písmenem, avšak aby bylo rozpoznání
    zjednodušeno (přesněji řečeno, aby stačilo jedno porovnání), tak kód pro
    binární příkazy je záporný. Firmware přečte kód příkazu jako 8-bitové
    číslo se znaménkem. Pokud je záporné, tak automaticky předpokládá, že vše
    co následuje je binární příkaz. Protože data v příkazu nejsou nijak
    kódována, tak není např. možné předpokládat, že se v příkazu neobjeví číslo
    10, které v ASCII tabulce reprezentuje znak konce řádku. Z~toho důvodu jsou
    příkazy ukončeny bytem 0.

    Odpovědi v binárním režimu začínají hlavičkou, která obsahuje identifikaci
    a~velikost následující odpovědi. Díky tomu nemusí být nějak speciálně
    ukončovány.

    Hlavní výhodou binárního režimu tedy je úspora dat při přenosu a možnost
    vyhnout se kódování odpovědi a posílat ji jako surová data.

    \subsection{Provedené úpravy}
    \label{btcom:upravy}

    Provedené úpravy se snaží vyřešit problémy stávajícího řešení a také
    umožňují použití asynchronní komunikace. Ke každému příkazu poslanému
    robotovi se přidává časový otisk, jedná se o 8--bitové číslo, které určuje
    pořadí příkazu v~sérii. Ve své odpovědi pak robot uvede jak kód označující
    typ příkazu, na který odpovídá, tak také došlý otisk. Díky tomu je možné
    téměř jednoznačně (otisk se samozřejmě opakuje po určité době znova) svázat
    odpověď se zaslaným příkazem. Pokud tedy nějaké číslo chybí v posloupnosti
    přijatých odpovědí, je jasné, že tato odpověď nedorazila.

    Odpovědi na binární příkazy nyní vždy obsahují hlavičku, která je pro
    všechny stejná a obsahuje samozřejmě kód příkazu, na který odpovídá, a také
    jeho timestamp, ale hlavně obsahuje i velikost dat, která budou následovat.
    Při binární komunikaci totiž není možné rozpoznat konec přenosu dat. V
    původní verzi firmware se počítalo s tím, že po každém odeslaném příkazu
    bude uživatel čekat na jeho odpověď a tedy bude vědět jakého formátu jsou
    data, která očekává. Při asynchronní komunikaci bychom neměli o příchozích
    datech nic předpokládat. Pokud známe jejich velikost, můžeme je nejprve
    načíst a až pak hledat příkaz, kterému patří.

    Do firmware byly také přidány nové příkazy. Robot obsahuje 3 mikrofony, ale
    není jednoduché nasnímaná data odeslat do PC kvůli jejich velikosti. Pro
    programy, které jsou spouštěny přímo v robotovi, ovšem existuje knihovna
    umožňující provádění FFT (rychlé Fourierovy transformace). Ta byla přidána
    do BTcomDM firmwaru a díky ní je možné jedním příkazem zapnout snímání dat
    z mikrofonu, na nasnímaných datech provést transformaci a odeslat je do PC.

    Dalším problémem, se kterým se musí uživatel e-puck robota potýkat, je
    kapacita baterie a nepřítomnost jakéhokoli ukazatele jejího stavu. Tento
    problém byl alespoň částečně vyřešen, do kontrolního programu v robotovi
    bylo přidáno přerušení, které je aktivováno právě tehdy, když napětí na
    baterii již není dostatečné pro bezproblémový provoz. Při vyvolání tohoto
    přerušení jsou zapnuty všechny LED na obvodu robota a tak uživatel dostane
    vizuální upozornění, že je třeba baterii vyměnit.

    \section{Synchronní komunikace}
    \label{btcom:sync}

    Hlavním cílem synchronní komunikace je umožnit uživateli knihovny
    jednoduché zasílání příkazů, u kterého není kladen důraz na ošetřování chyb
    spojení. Počítá se s tím, že uživatel bude využívat synchronní komunikaci
    pro experimentování s~robotem a interaktivní ovládání.

    Hlavní výhodou synchronní komunikace je její jednoduchost. Jak lze vidět
    ve zdrojovém kódu \ref{lst:sync_example}, příkaz lze předat robotovi
    zavoláním pouze jedné metody, která navíc i vrátí odpověď od robota. V
    tomto případě se jedná o~získání rychlosti motorů. Díky tomu lze robota
    ovládat i z interpretu Pythonu, čímž se výrazně zkrátí doba potřebná pro
    testování nových nápadů. Tam, kde je normálně potřeba napsat zdrojový kód,
    zkompilovat aplikaci a následně ji pustit, tady stačí pouze pustit Python,
    importovat knihovnu, vytvořit spojení (což jsou 2 řádky kódu) a~pak už nám
    nic nebrání v posílání příkazů.

    Každé zavolání příkazu odešle robotovi zprávu (bližší popis zprávy v části
    \ref{btcom:prikazy}) a následně zablokuje (zavoláním metody {\tt read} na
    sériové spojení) další vykonávání příkazů, dokud neobdrží odpověď. Uživatel
    si tedy musí sám dávat pozor, zda-li se mu nějaký příkaz neztratil. Na
    druhou stranu hlavní výhodou synchronní komunikace je, že uživatel vždy
    přesně ví, co se děje.

    \begin{listing}[H]
    \begin{pyc}
>>> from epuck import Controller
>>> controller = Controller('/dev/rfcomm0')
>>> controller.get_speed()
(0,0)
    \end{pyc}
    \caption{Příklad synchronní komunikace}
    \label{lst:sync_example}
    \end{listing}

    \section{Asynchronní komunikace}
    \label{async-impl}

    Asynchronní komunikace se snaží co nejvíce usnadnit ovládání robota, skrýt
    některé složitější mechanismy komunikace před uživatelem a přitom zajistit
    spolehlivou komunikaci. Krátká ukázka asynchronní komunikace je ve
    zdrojovém kódu \ref{lst:async_example}.

    \begin{listing}[H]
    \begin{pyc}
>>> from epuck import Controller
>>> controller = Controller('/dev/rfcomm0', asynchronous=True)
>>> request = controller.get_speed()
>>> request.get_response()
(0,0)
    \end{pyc}
    \caption{Příklad asynchronní komunikace}
    \label{lst:async_example}
    \end{listing}

    Základním požadavkem asynchronní komunikace je oddělení volání příkazu od
    jeho provedení. Provedení totiž může trvat delší dobu a to se může ukázat v
    některých aplikacích jako klíčový problém. Příkazy tedy nečekají na
    odpověď, ovšem jak se pak uživatel dostane k odpovědi? Zavolání příkazu
    nemůže vrátit odpověď, ale na druhou stranu může vrátit uživateli odkaz na
    objekt s potenciální odpovědí. Díky tomu se pak může tázat, zda-li odpověď
    přišla a nakonec ji také i~získat. Ukázka je ve zdrojovém kódu
    \ref{lst:oddeleni}. Pro vysvětlení, jak tento odkaz funguje, bude potřeba
    osvětlit vnitřnosti asynchronní komunikace.

    \begin{listing}[H]
    \begin{pyc}
# Zaslani prikazu
request = controller.get_speed()
# Kontrola, zda-li uz prisla odpoved
while not request.response_received():
    # Program zde muze delat cokoli jineho,
    # zatimco ceka na odpoved
# Ziskani odpovedi
value = request.get_response()
    \end{pyc}
    \caption{Ukázka oddělení zaslání příkazu a vyzvednutí odpovědi}
    \label{lst:oddeleni}
    \end{listing}

    Asynchronní komunikace už od začátku spojení s robotem probíhá v
    samostatném vlákně, které je nezávislé od vlákna, ve kterém běží program
    uživatele. Toto komunikační vlákno zastává tři funkce:

    \begin{enumerate}
        \item odesílání příkazů,
        \item zpracování příchozí komunikace a
        \item kontrola chyb.
    \end{enumerate}

    Příchozí komunikace je získávána pomocí sériového připojení vytvořeného
    knihovnou pySerial \cite{pyserial}. Připojení podporuje standardní metody
    pro práci se sokety a je tedy možné používat funkci select operačního
    systému, která dokáže kontrolovat více připojení a oznámit přijetí dat na
    libovolném z nich. První připojení, které je kontrolováno, tedy je sériové
    připojení s robotem. Další spojení slouží k~získávání nových příkazů.

    \subsection{Odeslání příkazu}

    Při odesílání je příkaz zavolán ve vláknu s uživatelským programem, ovšem
    samotná zpráva robotovi musí být odeslána z vlákna asynchronní komunikace.
    V~opačném případě by uživatelský program byl zpomalen přenosem k~robotovi,
    navíc asynchronní vlákno má větší přehled o spojení s robotem a tedy např.
    nedojde ke kolizi mezi posíláním a přijímáním dat. Je tedy nutné předat
    informaci o novém příkazu mezi vlákny.

    V uživatelském vláknu je příkaz zařazen do fronty pro odchozí příkazy,
    asynchronní vlákno je pak následovně informováno o novém příkazu, nakonec
    je vytvořen odkaz na příkaz, který je vrácen uživateli. Žádná z těchto
    instrukcí není blokující, tedy odeslání příkazu nestojí uživatele téměř
    nic.

    Dokončení odesílání už probíhá v asynchronním vláknu. Funkce select je
    přerušena informací o novém příkazu. Následně je příkaz odebrán z fronty,
    odeslán robotovi a zařazen do fronty odeslaných zpráv. Zde čeká na příchozí
    odpověď. Uživatel stále vlastní odkaz na tento příkaz, i když se přesunul.

    \subsection{Zpracování odpovědi}

    Pokud se robot rozhodne odeslat odpověď na příkaz, tak se to asynchronní
    vlákno dozví díky funkci select, která označí spojení, připravené pro čtení.
    Následuje tedy přečtení a zpracování odpovědi. Nejprve je potřeba
    rozhodnout, zda-li se jedná o binární nebo textovou odpověď. Na to stačí
    přečíst první znak, ten určuje, na který příkaz robot odpovídá. Další znak
    určí timestamp, díky němu už je možné jednoznačně najít příkaz, kterému
    odpověď patří, ve frontě odeslaných.

    Odpověď pak už je pouze předána příkazu, který si ji uloží. Příkaz je
    následně odebrán z fronty odeslaných příkazů. Pokud si uživatel uložil
    odkaz na tento příkaz, tak stále má možnost získat odpověď. V opačném
    případě už na objekt nic neukazuje a tedy bude smazán.

    Dost často je třeba odpověď ihned použít pro další zpracování a následující
    příkazy na ní nezáleží. Pak se zdá nutnost neustále kontrolovat, zda už od
    robota něco přišlo, nepřirozená. Lepší možnost je říct knihovně, co se má
    vykonat po přijetí odpovědi a nechat ji, ať se o vše postará sama. Z~tohoto
    důvodu má každý příkaz navíc parametr callback, přes který se může knihovně
    předat funkce, která bude spuštěna po přijetí odpovědi. Ukázka je ve
    zdrojovém kódu \ref{lst:callback}.

    \begin{listing}
    \begin{pyc}
>>> def zpracovani_rychlosti(vysledek):
...    leva, prava = vysledek
...    # Prace s vysledkem
...
>>> controller.get_speed(callback=zpracovani_rychlosti)
    \end{pyc}
    \caption{Ukázka použití callback funkce}
    \label{lst:callback}
    \end{listing}

    Při psaní callback funkcí je důležité pamatovat na to, že knihovna volá
    tyto metody ve stejném vlákně, v jakém provádí vše ostatní. Proto by
    callback funkce neměla být výpočtově náročná. Popřípadě by si měla pro
    složitější výpočty vytvořit vlastní vlákno.

    \subsection{Časový limit}

    Může se stát, že příkaz k robotovi nedorazí. Tato skutečnost se těžko
    rozpoznává. Robotovi zabírá vykonání příkazu delší dobu, anebo o příkazu
    vůbec neví? Kdy už se má příkaz považovat za ztracený? Pro vyřešení těchto
    problémů slouží časový limit. Pro každý příkaz má robot přidělen časový
    limit, v kterém musí odpovědět. Časový limit je vždy předán funkci select,
    která vlákno zablokuje právě na tuto dobu. Po uplynutí dojde ke kontrole
    stáří příkazů. Každý příkaz, který už je ve frontě příliš dlouho, je
    označen za ztracený a poslán znova. Je omezen počet pokusů o zaslání
    příkazu, po jejich vyčerpání je uživateli oznámena chyba (vyhozena
    výjimka). Robot je v takovém případě nejspíše neovladatelný.

    Časový limit je jednotný pro všechny příkazy a nastavuje se při vytváření
    instance objektu \code{Controller} (příklad ve zdrojovém kódu
    \ref{lst:timeout}). Je možné nastavit dva parametry:

    \begin{itemize}
        \item timeout -- časový limit v sekundách (je možné zadat i desetinné číslo),
        \item max\_tries -- maximální počet pokusů o zaslání příkazu před
        oznámením chyby uživateli.
    \end{itemize}

    Pokud uživatel tyto hodnoty nezmění, tak v základním nastavení je časový
    limit pro každý příkaz půl sekundy a je povoleno až 10 opakování.

    \begin{listing}
        \begin{pyc}
c = Controller('/dev/rfcomm0',
               asynchronous=True,
               timeout=0.5,
               max_tries=5)
        \end{pyc}
        \caption{Nastavení časového limitu a maximálního počtu pokusů}
        \label{lst:timeout}
    \end{listing}

    \section{Ovládání robota}
    \label{controller}

    Pokud je zajištěna komunikace s robotem, tak samotné jeho ovládání už je
    velmi jednoduché. Při vytváření ovladače robota musí uživatel specifikovat,
    zda-li chce komunikaci synchronní anebo asynchronní. Hlavní změna, kterou
    pozná, je v návratové hodnotě příkazů. Při synchronní komunikaci příkazy
    vrací rovnou výsledek, při asynchronní dostane nejprve odkaz na příkaz.
    Z~toho důvodu není možné psát programy, které budou fungovat se synchronní
    i asynchronní komunikací.

    Zaslání příkazu je pouhé zavolání funkce a předání správných parametrů.
    Ovladač se sám stará o~vytvoření příkazu, přidá k němu timestamp, zakóduje
    parametry a předá k odeslání.

    Stejně tak ovladač zpracovává příchozí odpovědi, než vydá uživateli
    výsledek. U jednoduchých příkazů, jako je třeba nastavení rychlosti motorů,
    jde pouze o~získání dvou čísel z řetězce. U složitějších příkazů už se může
    jednat o~netriviální činnost, například při získání fotky je potřeba
    převést fotku z~formátu, který používá robot, do formátu standardní
    knihovny Pythonu.

    \section{Výkon knihovny}
    \label{benchmark}

    Pro dálkové ovládání robota je důležité jak rychle můžeme reagovat na
    podněty z prostředí. Vyzkoušíme si tedy některé základní příkazy a budeme
    stopovat jak rychle se provádějí.

    Nejprve začneme zjištěním údajů ze senzorů překážek. Kód \ref{time1} (pouze
    relevantní část) měří kolikrát za sekundu je možné získat údaje z IR
    senzorů. Program jsem nechal běžet 500 sekund. Nejlepší výsledek bylo 50
    zavolání příkazu, nejhorší výsledek bylo jen 30 zavolání příkazu, v~průměru
    bylo posláno 39 příkazů. To je tedy v průměru 25ms na jednu iteraci.

\begin{listing}
\begin{pyc}
start_time = time.time()
count = [0]
while True:
    prox = c.get_proximity_sensors().get_response()
    count[-1] += 1
    if time.time() - start_time > 1:
        print count
        count.append(0)
        start_time = time.time()
\end{pyc}
\caption{Stopování rychlosti senzorů překážek}
\label{time1}
\end{listing}

    Zkusíme, jak moc se od sebe liší různé příkazy. V kódu \ref{time2} budeme
    získávat fotku z kamery. Pro barevné fotky s rozměry 40x40 je možné za
    sekundu v~průměru získat 3 snímky. Tady musíme započítat i převod z módu
    RGB565 do formátu používaného knihovnou PIL. Pokud nám bude stačit pouze
    černobílý obrázek stejných rozměrů, tak za sekundu jsme schopni jich získat
    v~průměru 5.

\begin{listing}
\begin{pyc}
start_time = time.time()
count = [0]
while True:
    c.get_photo().get_response()
    count[-1] += 1
    if time.time() - start_time > 1:
        print count
        count.append(0)
        start_time = time.time()
        \end{pyc}
\caption{Stopování rychlosti kamery}
\label{time2}
\end{listing}

    Stačilo nám tedy jen pár pokusů abychom zjistili, že možný počet zaslaných
    příkazů za sekundu se dramaticky liší v závislosti na typu příkazu anebo
    jeho parametrech. Důležité je také si uvědomit, že při asynchronní
    komunikaci po zavolání příkazu nemusíme čekat na odpověď a můžeme tedy
    ihned volat další příkaz. To s sebou nese jedno riziko, pokud budeme
    posílat nové příkazy rychleji, než je stíhá robot vykonávat, tak nám může
    přetéct fronta příkazů, která není nekonečná. Každé zavolání příkazu, který
    potřebuje načíst data, nám tuto frontu vyprázdní, protože se vždy musejí
    provést všechny předchozí příkazy a teprve pak je možné získat data, ale i
    tak je dobrým zvykem omezit maximální rychlost smyčky programu krátkou
    pauzou.
