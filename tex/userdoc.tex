\chapter{Uživatelská dokumentace}
\label{dokumentace}

    Interakce uživatele s knihovnou se skládá ze dvou částí. V té první se
    jedná o~připojení robota k počítači a uvedení jej do provozuschopného
    stavu. Součástí této inicializace je vytvoření spojení mezi počítačem a
    robotem a nainstalování firmware do robota. Vysvětlení jak se vše provádí
    je k nalezení v kapitole \ref{pripojeni_robota}.

    V druhém kroku už je robot připojený a tedy už mu můžeme posílat příkazy.
    Předchozí kapitola uvádí ukázky komunikace s robotem, ale až v kapitole
    \ref{ukazkovy_program} nalezne čitatel ucelený zdrojový kód, který se stará
    o ovládání robota od vytvoření prvotního spojení až po zastavení motorů po
    skončení programu.

    \section{Připojení robota}
    \label{pripojeni_robota}

    Pro využívání knihovny je nutné mít robota připojeného k PC. Robot se chová
    jako standardní Bluetooth zařízení. Poskytuje pouze jedinou službu a to
    sériovou komunikaci. Je tedy potřeba nakonfigurovat sériový port, přes
    který bude knihovna s robotem komunikovat.

    \subsection{Příprava}

    E-puck robot s počítačem komunikuje pomocí Bluetooth. Pro připojení robota
    k počítači na operačním systému Linux je tedy třeba mít nainstalovánu
    podporu pro Bluetooth. Většinou jsou to balíčky {\tt bluez-firmware} a {\tt
    bluez-utils}, dále je potřeba mít software pro správu PIN hesel ({\tt
    bluez-pin, bluez-gnome a~bluetooth-applet, \ldots}).

    Nejprve je potřeba zjistit adresu robota. Lze využít utilitu {\tt hcitool}:

    \begin{verbatim}
    $ hcitool scan
    Scanning ...
        10:00:E8:52:C6:3E	e-puck_1055
    \end{verbatim}

    Adresa robota se použije při nastavení virtuálního sériového portu v
    souboru {\tt /etc/bluetooth/rfcomm.conf}:

    \begin{verbatim}
    rfcomm0 {
        bind yes;
        device 10:00:E8:52:C6:3E; # Adresa zařízení
        channel 1;
        comment "e-puck_1055";    # Vlastní komentář
    }
    \end{verbatim}

    Tím jsme nakonfigurovali zařízení {\tt /dev/rfcomm0}.

    \subsection{Připojení}

    Zařízení reprezentující spojení s robotem jsme nakonfigurovali, nyní zbývá
    připojit robota:

    \begin{verbatim}
    $ sudo rfcomm connect rfcomm0
    Connected /dev/rfcomm0 to 10:00:E8:52:C6:3E on channel 1
    Press CTRL-C for hangup
    \end{verbatim}

    Je důležité, aby uživatel, který bude ovládat robota, měl právo přístupu
    k~zařízení \code{/dev/rfcomm0}.

    Po těchto příkazech by už mělo být vytvořené spojení s robotem a je možné
    mu začít posílat příkazy pomocí knihovny. V průběhu připojení se systém
    může dotázat na PIN, který je potřeba pro spárování zařízení pomocí
    Bluetooth.

    \subsection{Přehled možných chyb}
    Při připojení se můžou vyskytnou chyby, nejčastější jsou tyto:
    \begin{itemize}
    \item{{\em Can't connect RFCOMM socket: Connection refused} -- Nejspíše
        neběží žádná aplikace, která by se na heslo zeptala.}
    \item{{\em Can't open RFCOMM device: Permission denied} -- Uživatel nemůže
        přistupovat k zařízení \code{/dev/rfcomm0}.}
    \item{{\em Can't create RFCOMM TTY: Address already in use} -- Možná
        příčina je už připojené zařízení. Buď už běží {\tt rfcomm}, anebo jiná
        aplikace, která s~robotem udržuje komunikaci. Řešením je zavřít
        aplikace komunikující s robotem, popř. zkusit následující příkazy:
    \begin{verbatim}
    $ rfcomm release rfcomm0
    $ rfcomm connect rfcomm0
    \end{verbatim}
    }
    \end{itemize}

    \subsection{Nahrání firmware BTcomDM}

    Knihovna komunikuje s E-puck robotem pomocí posílání textových příkazů. Je
    třeba do robota nahrát firmware BTcomDM. Ten je dodáván s knihovnou.
    K jeho nahrání slouží skript \code{epuckupload} \cite{epuckupload}. Stačí
    se řídit pomocí \code{README} a do robota nahrát soubor \code{BTcomDM.hex}.
    Příklad použití:

    \begin{verbatim}
    $ epuckupload -f BTcomDM.hex rfcomm0
    \end{verbatim}

    Po spuštění příkazu je třeba robota restartovat (modré tlačítko na horní
    straně). Teprve pak se začne nahrávat nový firmware.

    \subsection{Instalace knihovny}

    Knihovna je dodávána v balíčku v příloze. Podporuje standardní instalaci,
    takže je možné ji nainstalovat pomocí skriptu \code{setup.py}.

    \begin{verbatim}
    $ tar -xvf epuck-0.9.1.tar.gz
    $ cd epuck-0.9.1/
    $ sudo python setup.py install
    \end{verbatim}

    Také je možné ji nainstalovat z webového repositáře PyPI (Python Package
    Index) \cite{pypi} pomocí programu \code{easy\_install}.

    \begin{verbatim}
    $ easy_install epuck
    \end{verbatim}

\section{Ukázkový program}
\label{ukazkovy_program}

    Pro ovládání robota slouží modul {\tt epuck} a objekt {\tt Controller}.
    Ovládání robota je velmi jednoduché. Ukážeme si triviální program, který
    přikáže robotovi, aby jel dopředu. Zároveň ale bude kontrolovat, zda-li
    před robotem není žádná překážka. Pokud by měl robot narazit, tak jej
    zastaví a skončí. Na tomto programu rozebereme jednotlivé příkazy:

\begin{mylisting}
\begin{pyc}
#!/usr/bin/env python
# -*- coding: utf-8 -*-

"""Go forward, stop in front of an obstacle."""

import time

from epuck import Controller

# Input larger than the threshold means there is an obstacle.
threshold = 300

# Create the controller. Robot is connected to /dev/rfcomm0
# and the communication will be asynchronous.
controller = Controller('/dev/rfcomm0', asynchronous=True)

# Set the speed of left and right wheel to 100.
controller.set_speed(100, 100)

# Ask for the values of proximity sensors
sensor_request = controller.get_proximity_sensors()
# Wait until the robot returns them.
sensor_values = sensor_request.get_response()

# Continue while there is nothing in front of the front left
# and front right sensor.
while (sensor_values['L10'] < threshold) \
       and (sensor_values['R10'] < threshold):
    # Wait for a while
    time.sleep(0.1)
    # Read new values
    sensor_request = controller.get_proximity_sensors()
    sensor_values = sensor_request.get_response()

# Stop the robot
controller.set_speed(0, 0)
\end{pyc}
\captionof{listing}{Jednoduchý asynchronní program pro ovládání robota}
\label{lst:simple_example}
\end{mylisting}

\subsection{Vysvětlení zdrojového kódu}

    Zdrojový kód \ref{lst:simple_example} tvoří celý program pro ovládání
    robota. Rozebereme si jej tedy řádek po řádku, abychom věděli, co která
    část dělá. První věcí, kterou je třeba vždy učinit, je import knihovny:

    \begin{pyc*}{firstnumber=8}
from epuck import Controller
    \end{pyc*}

    Nyní máme importovanou třídu {\tt Controller}, která slouží ke komunikaci
    s~robotem. Stačí už jen vytvořit samotné spojení:

    \begin{pyc*}{firstnumber=15}
controller = Controller('/dev/rfcomm0', asynchronous=True)
    \end{pyc*}

    Vytvoření spojení je jednoduché, první parametr je cesta k připojenému
    robotovi (jak připojit robota je popsáno v kapitole
    \ref{pripojeni_robota}). Druhý parametr je nepovinný a určuje, zda-li má
    komunikace probíhat asynchronně. V tomto případě chceme asynchronní
    komunikaci, pokud bychom druhý parametr vynechali, tak by komunikace byla
    synchronní.

    Nyní už můžeme zadávat příkazy:

    \begin{pyc*}{firstnumber=18}
controller.set_speed(100, 100)
    \end{pyc*}

    Zde nastavíme rychlost robota na 100 tiků za sekundu. V tomto případě nám
    stačí pouze vykonat příkaz a nepotřebujeme získat odpověď od robota. V
    případě změny rychlosti robot pouze pošle potvrzení, protože používáme
    asynchronní komunikaci, toto potvrzení bude zpracováno automaticky.

    Pokud ovšem chceme i odpověď, tak se příkaz skládá ze dvou kroků:

    \begin{pyc*}{firstnumber=20}
# Ask for the values of proximity sensors
sensor_request = controller.get_proximity_sensors()
# Wait until the robot returns them.
sensor_values = sensor_request.get_response()
    \end{pyc*}

    Nejprve pošleme příkaz a uložíme si návratovou hodnotu. Ta reprezentuje
    odpověď, která přijde na zaslaný příkaz. Druhým krokem pak je získání této
    odpovědi.

    Když už jsme získali nějaké data od robota, tak je můžeme zpracovat (v
    tomto případě to jsou hodnoty z IR senzorů):

    \begin{pyc*}{firstnumber=27}
while (sensor_values['L10'] < threshold) \
       and (sensor_values['R10'] < threshold):
    \end{pyc*}

    Kontrolujeme zde, zda-li  robot nestojí před překážkou. Bližší popis je v
    dokumentaci metod. V případě, že před robotem nic není, můžeme program na
    chvíli uspat, zatímco se robot posune o kus dál a my můžeme znova
    zkontrolovat senzory:

    \begin{pyc*}{firstnumber=29}
    # Wait for a while
    time.sleep(0.1)
    # Read new values
    sensor_request = controller.get_proximity_sensors()
    sensor_values = sensor_request.get_response()
    \end{pyc*}

    Vzhledem k tomu, že kontrola probíhá v nekonečné smyčce tak krátká pauza
    slouží k nevytěžování procesoru PC na 100\% a k zabránění zahlcení robota
    příkazy.

    Pokud narazí robot na překážku, tak jediné, co mu zbývá před koncem
    programu, je zastavit motory:

    \begin{pyc*}{firstnumber=36}
    controller.set_speed(0, 0)
    \end{pyc*}

    \subsection{Synchronní verze}

    Nyní si ukážeme, jak by vypadal stejný program napsaný synchronně (zdrojový
    kód \ref{lst:simple_sync_example}).

    \begin{mylisting}
    \begin{pyc}
#!/usr/bin/env python
# -*- coding: utf-8 -*-

"""Go forward, stop in front of an obstacle."""

import time

from epuck import Controller

# Input larger than the threshold means there is an obstacle.
threshold = 300

# Create the controller. Robot is connected to /dev/rfcomm0
# and the communication will be asynchronous.
controller = Controller('/dev/rfcomm0', asynchronous=False)

# Set the speed of left and right wheel to 100.
controller.set_speed(100, 100)

# Ask for the values of proximity sensors
sensor_values = controller.get_proximity_sensors()

# Continue while there is nothing in front of the front left
# and front right sensor.
while (sensor_values['L10'] < threshold) \
    and (sensor_values['R10'] < threshold):
    # Wait for a while
    time.sleep(0.1)
    # Read new values
    sensor_values = controller.get_proximity_sensors()

# Stop the robot
controller.set_speed(0, 0)
\end{pyc}
\captionof{listing}{Jednoduchý synchronní program pro ovládání robota}
\label{lst:simple_sync_example}
\end{mylisting}
\hfill\\

Prvním rozdílem je nastavení synchronní komunikace na řádku 15. Dále pak vždy
při získávání dat z robota můžeme vynechat jeden řádek a nechat pouze volání
příkazu, které rovnou vrací výsledek (řádky 21 a 30).

Další příklady kontrolních programů jsou k nalezení v kapitole \ref{priklady}.
