\chapter*{Úvod}
\addcontentsline{toc}{chapter}{Úvod}

Slovo robot pochází od českého spisovatele Karla Čapka, který jej poprvé
použil ve své divadelní hře R.U.R. V češtině už po staletí existovalo slovo
robota, které znamená nucenou práci. Roboti jsou nyní našimi neocenitelnými
pomocníky, zatím nejčastěji ve formě průmyslových, anebo medicínských robotů,
kteří provádějí delikátní práce s neuvěřitelnou přesností.

Dnes už kdokoli, kdo má zájem o robotiku, může získat vlastního robota.
Prodává se dokonce stavebnice Lego s řídící jednotkou a několika senzory,
díky které si robota mohou postavit i děti. Pro ty, kdo by chtěli robota už
hotového, existuje například e-puck robot. Jedná se o drobného robota,
obsahujícího velkou škálu senzorů. Díky nim je ideální pro zkoušení
rozličných experimentů (práce s kamerou, pohyb mezi překážkami atd.).

Cílem této práce je vytvořit uživatelsky přívětivou a intuitivní knihovnu v
programovacím jazyku Python, kterou budou moci použít studenti pro ovládání
e-puck robota. Tato knihovna jim umožní ovládat všechny části robota z PC
přes Bluetooth spojení z operačního systému Linux. Díky tomu budou moci
využít výkonu svého počítače pro vytváření programů, které by nebylo možné
uskutečnit v robotovi kvůli jeho výpočetním omezením.

Většina knihoven pro ovládání robota umožňuje pouze synchronní komunikaci.
Součástí této práce ovšem bude i možnost provádět příkazy asynchronně. Z
toho důvodu je součástí práce i firmware robota, který umožní uživateli
nejen využít asynchronní komunikaci, ale také napomůže zlepšení spolehlivosti
přenosu. Dále bude provedeno několik úprav, které rozšíří možnosti ovládání
robota (např. signalizace vybité baterie, anebo možnost získat hodnoty z
mikrofonů).

Práce by měla sloužit pro potřeby studentů zajímajících se o programování
e-puck robota. Pro seznámení s knihovnou a možnostmi robota tedy bude součástí
knihovny i několik ukázkových programů. Jejich účelem je ukázat jakým způsobem
je možné robota ovládat a co vše dokáže.

Knihovna bude mít velmi jednoduché rozhraní, díky kterému bude možné
ovládat robota bez složité inicializace, kupříkladu i interaktivně z
interpretu Pythonu. Také bude možné knihovnu používat ve spolupráci s
externími knihovnami. Výsledkem bude např. jednoduché vytváření grafických
aplikací, anebo zpracování obrázků z kamery pomocí externích nástrojů.

\section*{Přehled kapitol}

\subsection*{Teoretická část}
V první části práce jsou představeny podklady a teoretické informace
vázající se k tématu. Po přečtení této části by měly být jasné všechny
pojmy a čitatel by měl mít dostatečné informace o problematice programování
robota pro napsání vlastního programu.

V kapitole \ref{e-puck robot} je představen e-puck robot. Je zde popsán
účel robota, okolnosti jeho vzniku a také všechny senzory a akční členy.
Také se zde uživatel dozví jaké jsou možnosti programování robota.

Při dálkovém ovládání robota je velmi důležitá komunikace mezi ním a
počítačem, na kterém je spuštěn ovládací program. Kapitola \ref{sync/async}
představuje rozdíly mezi komunikací synchronní a asynchronní. Vysvětluje
jaké jsou výhody a nevýhody obou řešení a jak se s nimi e-puck knihovna
vyrovnává.

V kapitole \ref{existujici prace} čitatel zjistí, že tato práce není
jediným pokusem o knihovnu pro ovládání robota. Bude představeno několik
programů, které slouží k ovládání anebo simulaci robota. Také budou
představeny knihy, které mohou sloužit pro rozšíření informací uvedených v
této práci, anebo jako zdroj inspirace pro kontrolní programy.

Nakonec bude představena specifikace knihovny (část \ref{specifikace}).
Čitatel se dozví jakými problémy trpěl původní firmware a jaké jsou nutné
změny, které bylo třeba vykonat. Dále jsou zde uvedena všechna vylepšení,
díky kterým se knihovna snaží soupeřit s konkurencí.

\subsection*{Praktická část}
V druhé části práce bude představena implementace knihovny. Bude
vysvětleno, jak byly použity mechanismy představené v teoretické části,
díky kterým může být komunikace asynchronní. Také budou představeny
přídavky ke standardním možnostem robota.

V kapitole \ref{btcom} je představen firmware BTcom, jak funguje posílání
příkazů robotovi a také zde jsou popsány všechny změny, které byly vykonány
na firmware robota.

Synchronní komunikace je předvedena v části \ref{btcom:sync} i s krátkou
ukázkou jak zasílat příkazy.

Asynchronní komunikace je druhým způsobem jak robotovi zasílat příkazy, je
o poznání složitější než synchronní komunikace a tedy je jí věnováno více
místa v části \ref{async-impl}. Čitatel se dozví co vše se děje na pozadí
při posílání asynchronních zpráv.

Jak funguje samotné ovládání robota je popsáno v části \ref{controller}.
Ovládání se liší podle komunikace. Při synchronní je transparentní a
jednoduché, při asynchronní je na druhou stranu spolehlivé, i když přináší
do komunikace jednu vrstvu navíc.

\subsection*{Příklady}

V části \ref{dokumentace} je návod pro práci s e-puck robotem. Ten začíná u
připojení robota k počítači a končí zasíláním příkazů. Také je zde uveden
ukázkový program a následně rozebrán řádek po řádku.

Následují příklady kontrolních programů v kapitole \ref{priklady}. Jde o
příklad ovládání robota a vyhýbání se překážkám (Braitenberg vehicle v
sekci \ref{braitenberg vehicle}), dále možnost změny programu pomocí
přepínače na robotovi (příklad \ref{LED}) a nakonec využití kamery a
externích knihoven pro detekci obličejů (příklad \ref{face detection}).

\subsection*{Přílohy}

Nakonec už následují jen přílohy. V příloze \ref{dokumentace api} čitatel
nalezne popis všech metod, které slouží pro ovládání robota. Jedná se o
popis všech příkazů, které je možné robotovi poslat.

Na přiloženém CD jsou k nalezení všechny zdrojové kódy knihovny, společně s
příklady a uživatelskou dokumentací.
