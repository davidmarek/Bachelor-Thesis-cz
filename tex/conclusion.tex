\chapter*{Závěr}
\addcontentsline{toc}{chapter}{Závěr}

    Ve své práci jsem splnil všechny cíle, které jsem si vytyčil. Vznikla
    knihovna, která je intuitivní, umožňuje rychlé prototypování ovládacích
    programů nebo interaktivní ovládání robota z interpretu. Na druhou stranu
    knihovna nabízí i podporu pro vytváření robustních aplikací, kde se můžeme
    spolehnout, že každý zaslaný příkaz bude proveden.

    Uživatel dostává možnost si vybrat mezi transparentní synchronní
    komunikací, nebo asynchronní komunikací, u které nemusí čekat na přijetí
    odpovědi. Tady se jedná o velkou výhodu pro vytváření aplikací, které mohou
    pracovat s došlými daty, zatímco čekají na další, nebo jde pouze o
    interaktivitu aplikace, kde není přípustné, aby docházelo k prodlevám,
    zatímco program čeká na odpověď.

    Dalším úspěchem je přepracování firmware BTcom, napsaného v jazyku C
    a~nahrávaného přímo do robota, který je standardně dodáván s e-puck
    robotem. S novou verzí BTcomDM je možné velmi přesně párovat zaslané
    příkazy a došlé odpovědi. Díky provedeným změnám bylo možné vytvořit
    kontrolní mechanismy zabraňující ztrátě příkazů. Každý příkaz má nyní danou
    časovou lhůtu, do které musí dojít odpověď, jinak je považován za ztracený
    a odeslán znova.

    Pokusil jsem se i vyřešit některé základní problémy, které provázejí e-puck
    robota. Situaci, kdy robot přestává reagovat, protože mu dochází baterie,
    byla vyřešena optickou signalizací. Jedná se na první pohled o drobnou
    změnu, ovšem pro programování robota jde o neocenitelnou pomůcku. Obzvlášť
    když si student robota půjčuje v počítačové laboratoři a tak si nikdy
    nemůže být jistý, kdy byl naposledy dobíjen.

    Pro uživatele knihoven neexistuje nic horšího, než když mají dokumentaci
    jednotlivých metod, ale nedočtou se jak tyto spojit dohromady a vytvořit
    tak ucelený program. Proto jsem jako součást práce napsal několik
    ukázkových programů, které ukazují jak snadné je knihovnu používat a
    integrovat s jinými knihovnami. Ovládání robota jde triviálně propojit s
    grafickým rozhraním pomocí knihovny Tkinter, s fotkami získanými z kamery
    robota jde pracovat pomocí Python Imaging Library (PIL) a díky tomu je
    možné použít libovolnou knihovnu pro práci s~obrázky (např. OpenCV pro
    počítačové vidění).

    Provedl jsem měření rychlosti knihovny. Zjistil jsem, že možný
    počet předaných příkazů za sekundu záleží na složitosti příkazu.
    Pro jednoduché příkazy bylo dosaženo rychlosti okolo 40 příkazů za sekundu.
    Za dostatečnou se považuje rychlost přes 10 příkazů za sekundu a tedy
    e-puck knihovna v~tomto ohledu obstála bez problémů. I přes přidané
    ověřovací podprogramy je rychlost nadprůměrná, což ukazuje, že knihovna je
    dobře použitelná.

    Možnosti dalšího rozvoje knihovny jsou v oblasti integrace se simulátory.
    Občas je výhodnější testovat nejprve kód v simulátoru a teprve pak jej
    přenést do robota. Nabízí se možnost přepsat komunikační vrstvu knihovny
    tak, aby příkazy neodesílala do robota, ale převáděla je na příkazy pro
    simulátor. Další možností je nad knihovnou postavit vyšší vrstu, která ji
    bude používat pro ovládání e-puck robota. Například projekt Pyro má
    jednotné rozhraní pro ovládání všech podporovaných robotů, ale e-puck zatím
    podporován není.
